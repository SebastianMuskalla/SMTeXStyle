\documentclass
[
    lang=de,
    folder=../style,
    a4paper,
    12pt,
    parskip=half-,
    numbers=enddot,
    fleqn,
]{../style/smexam}

\title{SMExam}
\subtitle{Vorlesung}
\author{Sebastian Muskalla}

\date{28. Juli 2020}
\examiner{Prof. Dr. Dozent}
\organizer{Mitarbeiter, M.Sc.}
\institution{Universität}
\term{Sommersemester 202X}

\numpages{4}

\begin{document}

    \begin{exam}

    %
    % Title box
    %

    \maketitle

    %
    % Instructions
    %

    {\small
    \begin{enumerate}
        \item \textbf{Bitte am Anfang ausfüllen:}
    %    \bigskip
        \begin{description}
            \item Vorname:
                \vspace*{0.3cm}
            \item Nachname:
                \vspace*{0.3cm}
            \item Matrikelnummer:
                \vspace*{0.3cm}
            \item Unterschrift:
                \vspace*{0.3cm}
        \end{description}
        \item
            Die Nummer Ihrer Klausur ist
            \quad
            \monobf{\# \currentnumber}\ .
            \quad
            Bitte merken Sie sich die Nummer.
            Wir werden Ihr Klausurergebnis anonymisiert unter Verwendung dieser Nummer bekanntgeben.
        \item
    %        Achten Sie darauf, dass Ihre Klausur vollständig ist und getackert bleibt~(\pageref{LastPage} Blätter)!
            Achten Sie darauf, dass Ihre Klausur \textbf{vollständig ist} und \textbf{getackert bleibt}~(23~Blätter).
        \item
            Benutzen Sie \textbf{nur das an dieses Blatt angeheftete Papier}.
            Bei Bedarf können wir weitere Leerblätter austeilen.
            Wenn der Platz auf der Vorderseite des jeweiligen Aufgabenblatts nicht ausreicht, \textbf{machen Sie kenntlich}, wo Sie die Bearbeitung der Aufgabe fortsetzen.
        \item
            Als Hilfsmittel sind \textbf{ausschließlich} Sprachwörterbücher sowie ein beidseitig \textbf{handschriftlich beschriebenes DIN A4-Blatt} erlaubt.
            Elektronische Geräte müssen während der Klausur ausgeschaltet bleiben.
            Täuschungsversuche werden als nicht bestanden gewertet und dem Prüfungsamt gemeldet.
        \item
            Schreiben Sie leserlich und bearbeiten Sie Ihre Klausur mit einem \textbf{dokumentenechten Stift} (nicht mit Bleistift, kein Tipp-Ex, kein Tintenkiller) und \textbf{nicht in roter oder grüner Farbe}.
        % \item
        %     Wir werden das Deckblatt während der Klausur auf korrekte Daten überprüfen. Legen Sie dazu Ihren \textbf{Studierendenausweis} und einen amtlichen Lichtbildausweis bereit.
        \item
            Wir werden die \textbf{Klausurergebnisse} auf unserer Website bekanntgeben:\\
            \mono{tcs.cs.tu-bs.de/teaching/TheoInf2Klausur\_WS\_20192020}\ .
        % \item
        %     Wir werden den Termin für die \textbf{Klausureinsicht} auf unserer Website bekanntgeben:\\
        %     \mono{tcs.cs.tu-bs.de/teaching/TheoInf2Klausur\_WS\_20192020}\ .
        \item
            Die \textbf{Bearbeitungszeit} beträgt \textbf{180 Minuten} (+ ggf. Zeit zum Lüften).
        \item
            Mit \textbf{40 Punkten} ist die Klausur \textbf{sicher bestanden.}
    \end{enumerate}
    }

    %
    % Grading
    %


    \begin{center}
        {\large{\textbf{Bepunktung:}}} (wird von den Korrektoren ausgefüllt)

        % \vspace*{-2mm}

        \begin{tabular}{|p{2cm}||c|c|c|c|c|c|c|c|c|c||c|}
            \hline
            \textbf{Aufgabe} & 1 & 2 & 3 & 4 & 5 & 6 & 7 & 8 & 9 & 10 & $\Sigma$ \\
            \hline
            \textbf{Max.}   & 10 & 10 & 10 & 10 & 10 & 10  & 10 & 10 & 10 & 10 & \textbf{100}\\
            \hline
            \textbf{Punkte} \newline \mbox{}& & & & & & & & & & & \\
            \hline
        \end{tabular}
    \end{center}


    \newpage

    \begin{exercise}{Aufgabe}[5 + 5 = 10]
        Beweisen Sie: $\NPTIME = \PTIME$.

        Außerdem: Toastbrot!
    \end{exercise}

    \newpage

    \begin{exercise}{Aufgabe ohne Punkte}
        Beweisen Sie: $\NPTIME = \PTIME$.

        Außerdem: Toastbrot!
    \end{exercise}

    \additionalpages{1}

    \end{exam}

\end{document}
